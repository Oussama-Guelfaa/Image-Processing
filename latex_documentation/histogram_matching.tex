Histogram matching (also known as histogram specification) is the transformation of an image so that its histogram matches a specified histogram. This is useful when you want to make the appearance of two images similar or when you want to enhance an image by specifying a histogram with the desired characteristics.

\\subsection{Theory}

The histogram matching method involves the following steps:
\\begin{enumerate}
\\item Compute the histogram and CDF of the input image.
\\item Compute the histogram and CDF of the reference histogram.
\\item For each intensity level in the input image, find the intensity level in the reference histogram that has the closest CDF value.
\\item Replace the intensity level in the input image with the corresponding intensity level from the reference histogram.
\\end{enumerate}

\\subsection{Implementation}

We have implemented histogram matching in the \\texttt{histogram\\_matching.py} module:

\\begin{verbatim}
def match_histogram_custom(image, reference_hist, bins=256):
    """
    Match the histogram of an image to a reference histogram.
    
    Args:
        image (ndarray): Input image (values between 0 and 1)
        reference_hist (ndarray): Reference histogram
        bins (int): Number of bins for the histogram (default: 256)
        
    Returns:
        ndarray: Image with matched histogram
    """
    # Compute the histogram of the input image
    hist, bin_edges = np.histogram(image.flatten(), bins=bins, range=(0, 1))
    
    # Compute the CDFs
    cdf_image = hist.cumsum() / hist.sum()
    cdf_reference = reference_hist.cumsum() / reference_hist.sum()
    
    # Create a lookup table for the mapping
    lookup_table = np.zeros(bins)
    for i in range(bins):
        # Find the intensity level in the reference histogram with the closest CDF
        lookup_table[i] = np.argmin(np.abs(cdf_reference - cdf_image[i]))
    
    # Normalize the lookup table to [0, 1]
    lookup_table = lookup_table / (bins - 1)
    
    # Apply the lookup table to the image
    bin_idx = np.digitize(image.flatten(), bin_edges[:-1])
    matched = lookup_table[bin_idx - 1]
    
    # Reshape back to the original image shape
    matched = matched.reshape(image.shape)
    
    return matched
\\end{verbatim}

We also provide a function to create a bimodal histogram as a reference:

\\begin{verbatim}
def create_bimodal_histogram(bins=256, peak1=0.25, peak2=0.75, 
                            sigma1=0.05, sigma2=0.05, 
                            weight1=0.5, weight2=0.5):
    """
    Create a bimodal histogram with two Gaussian peaks.
    
    Args:
        bins (int): Number of bins for the histogram (default: 256)
        peak1 (float): Position of the first peak (default: 0.25)
        peak2 (float): Position of the second peak (default: 0.75)
        sigma1 (float): Standard deviation of the first peak (default: 0.05)
        sigma2 (float): Standard deviation of the second peak (default: 0.05)
        weight1 (float): Weight of the first peak (default: 0.5)
        weight2 (float): Weight of the second peak (default: 0.5)
        
    Returns:
        tuple: (histogram, bin_centers)
    """
    # Create bin centers
    bin_centers = np.linspace(0, 1, bins)
    
    # Create the bimodal histogram
    hist = weight1 * np.exp(-((bin_centers - peak1) ** 2) / (2 * sigma1 ** 2))
    hist += weight2 * np.exp(-((bin_centers - peak2) ** 2) / (2 * sigma2 ** 2))
    
    # Normalize the histogram
    hist = hist / hist.sum()
    
    return hist, bin_centers
\\end{verbatim}

\\subsection{Command-Line Interface}

The histogram matching functionality is available through the command-line interface:

\\begin{verbatim}
# Apply histogram matching using the custom implementation
imgproc matching --method custom --peak1 0.3 --peak2 0.7 --image path/to/image.jpg

# Apply histogram matching using the built-in function
imgproc matching --method builtin --peak1 0.3 --peak2 0.7 --image path/to/image.jpg

# Apply both methods and compare
imgproc matching --method both --peak1 0.3 --peak2 0.7 --image path/to/image.jpg
\\end{verbatim}
